\documentclass{report}

\author{Patrick Lindsay}
\title{Software Engineering Notes}

\newcommand*{\glossaryname}{Dictionary}
\usepackage[nonumberlist]{glossaries}
\newcommand{\define}[2]{%
 \newglossaryentry{#1}{name=#1, description={#2}}%
 \glslink{#1}{}%
}
\makeglossaries

\begin{document}
\maketitle
\tableofcontents
\newpage
\part{Notes}
\chapter{January 15, 2012}
	\section{Overview}
	\underline{Software Engineering} - The application of sound engineering practices to software creation and maintenance.
		\subsection{Software development Life Cycle (Traditional Approach)}
			\begin{itemize}
				\item Requirements Phase
				\item Analysis or Specification Phase
				\item Design Phase
				\item Implementation/Integration Phase
				\item Maintenance Phase
				\item Retirement
			\end{itemize}
		\subsubsection{Requirements Phase}
			\begin{itemize}
				\item Determining the \textsc{needs} and \textsc{wants} of the client or customer.
				\item Determining the constraints of the system.
			\end{itemize}
		\subsubsection{Analysis or Specification Phase}
			\begin{itemize}
				\item After analyzing the requirements, construct a \emph{specification document} which explicitly describes what the product is to do, and the constraints under which it must operate.
				\item This includes the description of the input, output, actions, and UI.
				\item The specification document can be used as part of a contract with the client.
			\end{itemize}
			\paragraph{Problems with the Spec Document}
			\begin{enumerate}
				\item Ambiguity - one sentence may have more than one interpretation.
				\item Incompleteness - relevant fact or requirement is left out.
				\item Contradiction - two places in the spec document are in conflict.
			\end{enumerate}
		\subsubsection{Design Phase}
			\begin{itemize}
				\item Construct an \emph{Architectural Design}.
				\item Construct a \emph{Detailed Design}.
				\item Test for \emph{traceability}.
			\end{itemize}
			\begin{enumerate}
				\item \underline{Architectural Design} - Description of the product in terms of modules.
				\item \underline{Detailed Design} - Description of each module.
				\item \underline{Traceability} - each part of the design can be traced to a statement in the specification document.
			\end{enumerate}
		\subsubsection{Implementation Phase}
			\begin{itemize}
				\item Code each module from the detailed design.
				\item Programmer tests his/her own code separately.
				\item Modules are combined and tested by developers.
				\item Product is tested by SQA group. This is called product testing.
				\item Project is given to the client for acceptance testing.
			\end{itemize}
		\subsubsection{Maintenance Phase}
			\begin{itemize}
				\item Corrective Maintenance - bug squashing
				\item Enhancement Maintenance - Updates
				\begin{itemize}
					\item Perfective - client makes new demands
					\item Adaptive - changes in the environment of the product requires changes in the software.
				\end{itemize}
				\item Perform regression testing - insuring that changes have not affected already working functionality.
			\end{itemize}
		\subsubsection{Retirement Phase}
			\begin{itemize}
				\item Determining if desired changes are too costly.
				\item Determining if a product is obsolete.
			\end{itemize}
	\subsection{Four Components of the Software Engineering Enterprise}
	The four P's
	\begin{enumerate}
		\item Process
		\item Project
		\item People
		\item Product
	\end{enumerate}
		\subsubsection{Process}
			\begin{itemize}
				\item The process is sometimes called the life-cycle model or development sequence.
				\begin{itemize}
					\item Waterfall
					\item Spiral
					\item Incremental Build
				\end{itemize}
				\item Makes use of several process frameworks.
				\begin{itemize}
					\item Personal Software Process (PSP)
					\item Team Software Process (TSP)
					\item Capability Maturity Model (CMM)
				\end{itemize}
				\item Documentation Standards
				\begin{itemize}
					\item IEEE
					\item ANSI
				\end{itemize}
			\end{itemize}
		\subsubsection{Project}
			\begin{itemize}
				\item The set of activities needed to produce the required product.
				\item Project management is extremely important.
				\item Many projects are not about developing new products, but maintaining already existing \emph{legacy} systems.
			\end{itemize}
		\subsubsection{People}
			\begin{itemize}
				\item Team Organization
				\item Team Management
				\item Relationship with customer or client
				\item Relationship with end users
				\item Communication with upper management
			\end{itemize}
		\subsubsection{Product}
			Includes
			\begin{itemize}
				\item Requirement Specification Document
				\item Design Document
				\item Source Code
				\item Executable
				\item User Manuals
			\end{itemize}
\chapter{January 17, 2013}
	\section{Traditional Software Engineering Process}
		\subsection{Historical Influences}
			\begin{itemize}
				\item Structured programming (Edsger Dijkstra's letter calling "GOTOs" harmful) uses sequence control, iteration, invoking functions
				\item Object Oriented paradigm: the use of objects with data and functionality which can represent real-world entities.
			\end{itemize}
			\textit{Note:\\Silver Bullet ca. 1980s\\ Likened the software crisis to a werewolf.  Object Oriented Paradigm was the Silver Bullet.\\It did not work as expected.}
			\begin{itemize}
				\item Design Patterns: stock of reusable design elements (templates)
			\end{itemize}
		\subsection{Component Reuse}
			A component as defined by Meyer is " a program element satisfying:"
			\begin{enumerate}
				\item The element may be used by other program elements. (Clients)
				\item The clients and their authors do not need to be known to the element's author
			\end{enumerate}
		\subsection{Key Expectations of Software Engineering}
			\begin{enumerate}
				\item Decide in advance what the specific quality measures are to be for the project and product.\\
					\textit{Predetermine quantitative quality goals.}
				\item Gather data on all projects to form a basis for estimating future projects.
				\item All requirements, designs, code and test materials should be freely and easily available to all members of the team.\\
					\textit{Source code should always be available to all team members in an easily accessible and interpreted way.\\
								Git, Mercurial, etc.}
				\item A process should be followed by all team members. \textit{Uniformity}
					\begin{enumerate}
						\item Design only against requirements.
						\item Program only against design.
						\item Test only against requirements and design.\\
						\textsc{Always follow the recipe!}
					\end{enumerate}
				\item Measure and achieve quality goals.
			\end{enumerate}
	\section{Methods}
		\emph{Be able to draw and discuss these.}
		\subsection{Waterfall Method}
			\textsc{See diagram 52.9 on pg 53.}
			\begin{itemize}
				\item First described by William\textit{(?)} Royce in 1970.
				\item No phase is complete until documentation for that phase has been completed and approved by the SQA group.\\
					\textit{Very orderly; heavy on documentation.}
				\item Has been used with great success on a variety of products.
				\item Feedback loops permits modifications to be made to the previous phase.
			\end{itemize}
			\subsubsection{Advantages}
				\begin{enumerate}
					\item Enforced disciplined approach
					\item Requirement that documentation be provided at each phase.
					\item All products of the phase must be checked by SQA.
					\item Inherent aspect of each phase is testing.
				\end{enumerate}
			\subsubsection{Disadvantages}
				\begin{itemize}
					\item The resulting specification document may not be able to be understood by the client.
					\item It can lead to the construction of product that does not meet the client's needs.					
				\end{itemize}
		\subsection{Rapid Prototype Model}
			\textsc{See diagram on pg 55.}
			Construction of a functional subset of the desired product in order to allow the client and the developer to interact.\\
			\textit{Keyword is rapid.  This is a thrown-together, proof-of-concept type project; a mock-up.}
			\subsubsection{Advantages}
				\begin{itemize}
					\item The process is linear and possibly faster than the Waterfall Model
					\item Increases interaction between client and developer.
				\end{itemize}
			\subsubsection{Disadvantages}
				\begin{itemize}
					\item Client may inaccurately think the product is almost complete when viewing the prototype.
					\item Developer may attempt to use the prototype as part of the final product.
				\end{itemize}
		\subsection{Waterfall-Rapid Prototype Hybrid}
			May form a hybrid model using the rapid prototype as the first phase in the Waterfall Model in order to increase interaction but allow for feedback loops within the development of the product.
		\subsection{Incremental Model}
			Software is implemented, integrated, and tested as a series of incremental builds.\\
				\textit{Code pieces providing specific functions.}
			\subsubsection{Advantages}
				\begin{enumerate}
					\item Results in builds which can be developed in weeks, not months or years.
					\item End user need not learn the entire product at one time.
					\item Client need not pay for the entire product at one time.
					\item Developer gets paid earlier.\\ (At each build delivery)
					\item Open-ended design makes maintenance easier.
					\item Easier to make changes during development.
				\end{enumerate}
			\subsubsection{Disadvantages}
				\begin{enumerate}
					\item Each new build must fit in without destroying existing builds.\\
						\textit{Regression Testing}
					\item Requires more careful to design to make it open to additions.
					\item Can degenerate to a build and fix product if broken into too few builds.
				\end{enumerate}
		\subsection{Spiral Model}
			\textsc{See Figure 2.12 on pg 63 and Figure 2.13 on pg 65.}
			\begin{itemize}
				\item A Waterfall Model with each phase preceded by risk analysis in an attempt to control or resolve risk.
				\item Each phase is 360º.
				\item The measure of the radius is the cumulative cost to date.
				\item The measure of the angle is the progress measure.\\
					\textit{Each phase is 360º.\\
								Requires a very experienced engineer.}
			\end{itemize}
			\subsubsection{Advantages}
				\begin{enumerate}
					\item The emphasis on alternatives and constraints supports the reuse of existing software.
					\item The incorporation of software quality as a specific objective.
					\item Answers the question of how much testing should be performed in terms of risks.
					\item Maintenance is simply another cycle of the spiral, the same as development.
				\end{enumerate}
			\subsubsection{Disadvantages}
				\begin{enumerate}
					\item Intended exclusively for internal development.\\
						\textit{Client and developer are members of the same organization.}
					\item \emph{Applicable only to large-scale projects.}
					\item Must have developers who are skilled at pinpointing the possible risks.
				\end{enumerate}
		\section{Agile Methods}
			\subsection{General}
				According to the Agile Manifesto, they value:
				\begin{itemize}
					\item Individuals and interactions over processes and tools.
					\item Working software over comprehensive documentation.
					\item Customer collaboration over contract negotiation.
					\item Responsiveness to change over following a plan.\\
						\textsc{You don't always have to follow the recipe!}
				\end{itemize}
				\subsubsection{Traits}
					\begin{itemize}
						\item Highly iterative
						\item Pair programming with a focus on teamwork and ego-less programming.\\
							\textsc{This is mandatory.}
						\item Early and planned testing.
						\item Story cards \textit{Similar to storyboards in movies.}
						\item Refactoring \textit{Turning working code into better code.}
						\item Feedback
					\end{itemize}
				\subsubsection{Principles Behind the Agile Manifesto}
					\begin{enumerate}
						\item Our highest priority is to satisfy the customer through early and continuous delivery of valuable software.
						\item Welcome changing requirements, even late in development. Agile processes harness change for the customer's competitive disadvantage.
						\item Deliver working software frequently, from a couple of weeks to a couple of months, with a preference to the shorter timescale.
						\item Business people and developers must work together daily throughout the project.
						\item Build projects around motivated individuals. Give them the environment and support they need and trust them to get the job done.
						\item The most efficient and effective method of conveying information to and within a development team is face-to-face conversation.
						\item Working software is the primary measure of progress.
						\item Agile processes promote sustainable development. The sponsors, developers, and users should be able to maintain a constant pace indefinitely.
						\item Continuous attention to technical excellence and good design enhances agility.
						\item Simplicity - the art of maximizing the amount of  work not done - is essential.
						\item The best architectures, requirements, and designs emerge from self-organizing teams.
						\item At regular intervals, the team reflects on how to become more effective, then tunes and adjusts its behavior accordingly.
					\end{enumerate}
\end{document}